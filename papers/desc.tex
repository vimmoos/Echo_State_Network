                                % Created 2020-05-16 Sat 18:35
                                % Intended LaTeX compiler: pdflatex
\documentclass[11pt]{article}
\usepackage[utf8]{inputenc}
\usepackage[T1]{fontenc}
\usepackage{graphicx}
\usepackage{grffile}
\usepackage{longtable}
\usepackage{wrapfig}
\usepackage{rotating}
\usepackage[normalem]{ulem}
\usepackage{amsmath}
\usepackage{textcomp}
\usepackage{amssymb}
\usepackage{capt-of}
\usepackage{hyperref}
\date{\today}
\title{}
\hypersetup{
 pdfauthor={},
 pdftitle={},
 pdfkeywords={},
 pdfsubject={},
 pdfcreator={Emacs 26.3 (Org mode 9.1.9)},
 pdflang={English}}
\begin{document}

\tableofcontents

\section{GENERATING A DRUM RHYTHM}
\label{sec:org17a94f6}
\subsection{Initial thought}
\label{sec:org74fea72}
\begin{itemize}
\item maybe we can do some sort of hierarchical structure to handle
different level of patterns (like general pattern against specific
decoration pattern)
\item ESN or LSTM ?
maybe it's better to go for the ESN, simply because the LSTM's
methods for generating music have been already explored
\end{itemize}

\subsection{Papers}
\label{sec:orgdcea7bc}

\begin{itemize}
\item \href{./LSTM-generation-of-music.pdf}{./LSTM-generation-of-music.pdf} this paper show how to build
an LSTM network to generate classical music (piano)
\item \href{./2561\_Ivanchev2013.pdf}{./2561\(_{\text{Ivanchev2013.pdf}}\)} this paper used ESN for pattern
generation in robotics (maybe it could be useful to look at how
they design the network, etc)
\item \href{./anticipation-RNN.pdf}{./anticipation-RNN.pdf} this paper introduce a "new" type of
RNN architecture which generate music given some kind of "style"
constraint. (can be useful for some inspiration on the structure
of our NN)
\item \href{./functional-taxonomy-of-music-gen-sys.pdf}{./functional-taxonomy-of-music-gen-sys.pdf} quite general paper
about music generation. (the main point here can be how to
measure performance)
\item \href{./LSTM-music-generation-2020.pdf}{./LSTM-music-generation-2020.pdf} this paper show how to use LSTM
to generate music, it tries different kind of RNN.(look at loss
function and architecture)
\item \href{./PracticalESN.pdf}{./PracticalESN.pdf} an introduction on how to use and apply
Echo State Network (seems quite useful)
\item \href{./ESN-levenberg-for-chaotic-time-series-prediction.pdf}{./ESN-levenberg-for-chaotic-time-series-prediction.pdf}
\href{./ESN-levenberg-new-method.pdf}{./ESN-levenberg-new-method.pdf} these papers illustrate a
different technique to learn the output weights which instead of
using a simple linear regression uses the Levenberg-Marquardt
algorithm, it seems quite complicated math for now so maybe it's
better to go through when we have more knowledge.
\item \href{./ESN-opt-binary-grey-wolf.pdf}{./ESN-opt-binary-grey-wolf.pdf} this paper illustrate an hybrid
ESN network called BGWO-ESN which is based on the Grey wolf
optimization algorithm, it also compare this GWO to other
evolutionary algorithm like GA(genetic algorithm). it shows that
this new hybrid network has better results compared to standard
ESN and GA-ESN on financial prediction. (it seems interesting)
\item \href{./ESNTutorialRev.pdf}{./ESNTutorialRev.pdf} just a tutorial on RNN by Jaeger.
\item \href{./ESN-universal.pdf}{./ESN-universal.pdf} a proof of the universality of the ESN.
\item \href{./hierarchicalesn\_techrep10.pdf}{./hierarchicalesn\(_{\text{techrep10.pdf}}\)} an example of how to do
hierarchical ESN by Jaeger. we can get some ideas from there if
we go for an hierarchical architecture.
\item \href{./ESN-music-by-prediction.pdf}{./ESN-music-by-prediction.pdf} the bachelor thesis on nestor
about ESN and music generation. I didnt read it but it seems useful.
\end{itemize}
\end{document}
