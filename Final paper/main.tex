%% LyX 2.2.4 created this file.  For more info, see http://www.lyx.org/.
%% Do not edit unless you really know what you are doing.
\documentclass{extarticle}
\usepackage[utf8]{inputenc}
\usepackage{amsmath}
\usepackage{amsthm}
\usepackage[authoryear]{natbib}

\makeatletter
%%%%%%%%%%%%%%%%%%%%%%%%%%%%%% Textclass specific LaTeX commands.
\numberwithin{equation}{section}
\numberwithin{figure}{section}

%%%%%%%%%%%%%%%%%%%%%%%%%%%%%% User specified LaTeX commands.


\usepackage{graphicx}

\makeatother

\begin{document}
\begin{titlepage}
\begin{center}        
\vspace*{1cm}
       \textbf{ \huge Neural Networks Paper Title\\}
       \vspace{0.5cm}         Thesis Subtitle    \\                 
	   \vspace{1.5cm}
       \textbf{Sneha Lodha \\ Marco Gallo \\ Max Falziri \\ Yasmin de Groot}
       \vfill                                        
\vspace{0.8cm}              
                     
Neural Networks\\        
University of Groningen\\                
7th of July 2020                 
\end{center}
\end{titlepage}
\begin{abstract}
abstract text (10 lines)
\end{abstract}
\tableofcontents{}

\newpage

\section{Introduction}

\paragraph{Paragraph 1 }

paragraph text

\section{Data}

\subsection*{MIDI-File}

\subsection*{Generation}

\section{Methods}

\subsection*{Pre-processing}

- convert midi file into csv using https://www.fourmilab.ch/webtools/midicsv/ 

- parse csv into a signal where the length represents the duration
of the midi file. Each element is a vector of length max amount of
nodes and each value of this vector represents the intensity / 0/1
(testing) 

- generation which contains the same properties of the signals, but
it is generated instead of using midi files. 

\subsection*{Echo State Network}

input is the signal generated by the preprosessing. The parameters
are: The density of the resevoir. The spector radius. The leaking
rate. The size of the resevoir. Ridge regression (parameter). Transformer
(?). Output

\subsection*{Post-processing}

\subsection*{Fitting}

evaluation function (MLP/Jaeger idea). expanding parameters: bfs with
gradient decent.

\section{Results}

\section{Discussion}

\bibliographystyle{apalike}
\addcontentsline{toc}{section}{\refname}\bibliography{references}
 
\end{document}
