%% LyX 2.2.4 created this file.  For more info, see http://www.lyx.org/.
%% Do not edit unless you really know what you are doing.
\documentclass{extarticle}
\usepackage[utf8]{inputenc}
\usepackage{amsmath}
\usepackage{amsthm}
\usepackage[authoryear]{natbib}
\usepackage{graphicx}

\makeatletter

\numberwithin{equation}{section}
\numberwithin{figure}{section}

\makeatother

\begin{document}
\begin{titlepage}
\begin{center}        
\vspace*{1cm}
       \textbf{ \huge Neural Networks Paper Title\\}
       \vspace{0.5cm}         Thesis Subtitle    \\                 
	   \vspace{1.5cm}
       \textbf{Sneha Lodha \\ Marco Gallo \\ Max Falziri \\ Yasmin de Groot}
       \vfill                                        
\vspace{0.8cm}              
                     
Neural Networks\\        
University of Groningen\\                
7th of July 2020                 
\end{center}
\end{titlepage}

\begin{abstract}
abstract text (10 lines)
\end{abstract}
\tableofcontents{}
\newpage


\section{Introduction}

Classical music generation outside the domain of AI is mainly inspired by creativity and the knowledge of some musical elements.
There are several methods that have been used in the past to perform similar tasks, some of which include the use of algorithmic composition and Markov chains. %Ref?% 
Composing music through algorithmic rules can be restrictive in terms of originality, and user feedback. [1] Once the algorithms to generate the music 
have been defined, there is little to modify or fix. %%this sentence is weird%% 
When using Markov chains for the generation of music, although the presented output has more variation, 
drawbacks such as the violation of the Markov property come hand in hand. [2] This is mainly due to the fact that Markov property relies on simply the 
previous output which is not enough when it comes to generating musical patterns.  In this project, we will be attempting to generate musical drum 
patterns using an Echo State Network, also referred to as an ESN, and measure how close the generated pattern is to actual music.\\

An ESN is a type of Recurrent Neural Network (RNN) with an input layer, a hidden layer, also known as a reservoir and an output layer. The interesting 
aspect of the ESN is that the reservoir to output weights are the only ones that need to be learned, and the weights of the reservoir itself are fixed. 
Due to this property of the learning the ESN is considered faster than some other RNNs.
We decided to use  

\section{Data}

In this section the data used in this project is descibed and explained. The first subsection will explain how the drums are represented, and the second 
subsection will decribe the generation of the music.

\subsection*{Representation of the drums}

The drum beats are represented in the network in the form of a matrix. In Table 1 an example of such a matrix can be seen. 
All the matrices will have 9 colums, each representing different notes as seen in the table.
Each row in the matrix this matrix will represent a quarter, and 4 row will represent a bar. If the tempo of the song is 
for example in 8th, 8 rows will together represent a bar. This is thus dependent on the tempo of the song. 



\begin{table}[h!]
       \centering
       \resizebox{\columnwidth}{!}{%
       \begin{tabular}{l|lllllllll}
       \cline{2-10}
                                       & Hi-hat-closed & Hi-hat-open & Bass drum & Crash & Snare & High-tom & Mid-tom & Floor-tom & Ride \\ \hline
       \multicolumn{1}{|l|}{Quarter 1} & 0             & 0           & 1         & 0     & 0     & 0        & 0       & 0         & 0    \\
       \multicolumn{1}{|l|}{Quarter 2} & 0             & 0           & 0         & 0     & 0     & 0        & 0       & 0         & 0    \\
       \multicolumn{1}{|l|}{Quarter 3} & 0             & 0           & 1         & 0     & 0     & 0        & 0       & 0         & 0    \\
       \multicolumn{1}{|l|}{Quarter 4} & 0             & 0           & 0         & 0     & 0     & 0        & 0       & 0         & 0   
       \end{tabular}%
       }
       \end{table}

- pseudocode of the generator
- what happens with 8th and 16th 
- this is 1 bar, this matrix repeats for multiple bars
- how much we generated 

\section{Methods}

In this experiment, one pipeline is used. This consists of the learning
algorithm, the post processing and the performance metrics. All of these steps will
be explained in detail in this section. The reason there is no pre-process present
is because the data is generated in such a way that it does not need to be pre-processed 
for the network. The way this data is generated was explained in the previous section. 
The code of this project is mainly written in Python, while the statistical analysis 
is done in R. 

\subsection{Echo State Network}

In this music generating task an ESN is used. The input signal of
the ESN is created as described in the pre-prosessing. The set up
of this ESN is based on ``GUIDE REFERECE``. This guide contained
the base code of this project. It also provided the needed knowledge about the 
network and its parameters as descibed below. The given code has been rewritten 
to fit this project. The transformers, sparce matrix and noice vector were
added as an addition. The documentation of the code can be found {*}here{*}.
Regarding the ESN, the following parameters have been set:

%% Maybe add the system equations etc %%

\subsubsection*{The resevoir size}

Within ESNs, it is serverly important that the resevoir is big enough,
such that is it possible to obtain the target output $y^{target}(n)$
from a linear combination of this signal space. The resevoir in this
project has been set to ...

\subsubsection*{The resevoir density}

The density of a resevoir is mainly dependent on the distribution
of the nonzero elements in the resevoir. In this project a basic uniform
distribution is used. Besides the distribution, the density of the
resevoir is set to .... 

\subsubsection*{Spectral radius}

Another main parameter for fitting the ESN is the spectral
radius $\varrho$. This spectral radius is a parameter that will scale
the resevoir matrix $\mathbf{W}$. The effect this parameter has is
mainly seen on the learning accuracy of the the network. The new scaled
matrix $\mathbf{W}$ is calculated using 
\[
\boldsymbol{W{\scriptstyle new}}=\mathbf{W}*(\frac{\varrho}{\max(|\lambda|)})
\]

where $\lambda$ represents the eigenvalues of the resevoir matrix
$\mathbf{W}$, and $\mathbf{W_{new}}$ represents the updated resevoir
matrix. After experimenting, the $\varrho$ has been set to ...

\subsubsection*{Leaking rate}

The leaking rate $\alpha$ for an ESN determines how well a resevoir
unit maintains its value and how much it gets updated. Therefore $\alpha$
is one of the main parameters regarding the training process of this
project. In this project $\alpha$ is set to ...

\subsubsection*{Regularization}

The regularization of this project is implemented using a ridge regression 
in the learning step of the network. This is used to stabilize the output 
in the long run. The parameter that scales the identity matrix in the ridge
regression is the actual set parameter in this situation. In this code this 
parameter is set to $e^{-8}$.

Besides the ridge regression, a noise vector is also present in the code.
This is not used during this project. 

\subsubsection*{Transformers}

@Max :)
Transformer (the 3 mentioned is the intro paper: treshold, sigmoid,
sigmoid probability) (?) each transformer has a parameter and a squeezing
function (for now). 

\subsection{Post-processing}

\subsubsection*{Output}

not sure what to write here yet 

\subsection{Fitting}

evaluation function (MLP/Jaeger idea). expanding parameters: bfs with
gradient decent.

\section{Results}

\section{Discussion}

%% add bib %%
\end{document}
