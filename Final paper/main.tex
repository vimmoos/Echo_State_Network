%% LyX 2.2.4 created this file.  For more info, see http://www.lyx.org/.
%% Do not edit unless you really know what you are doing.
\documentclass{extarticle}
\usepackage[utf8]{inputenc}
\usepackage{amsmath}
\usepackage{amsthm}
\usepackage[authoryear]{natbib}

\makeatletter

%%%%%%%%%%%%%%%%%%%%%%%%%%%%%% LyX specific LaTeX commands.
\newcommand{\noun}[1]{\textsc{#1}}

%%%%%%%%%%%%%%%%%%%%%%%%%%%%%% Textclass specific LaTeX commands.
\numberwithin{equation}{section}
\numberwithin{figure}{section}

%%%%%%%%%%%%%%%%%%%%%%%%%%%%%% User specified LaTeX commands.


\usepackage{graphicx}

\makeatother

\begin{document}
\begin{titlepage}
\begin{center}        
\vspace*{1cm}
       \textbf{ \huge Neural Networks Paper Title\\}
       \vspace{0.5cm}         Thesis Subtitle    \\                 
	   \vspace{1.5cm}
       \textbf{Sneha Lodha \\ Marco Gallo \\ Max Falziri \\ Yasmin de Groot}
       \vfill                                        
\vspace{0.8cm}              
                     
Neural Networks\\        
University of Groningen\\                
7th of July 2020                 
\end{center}
\end{titlepage}
\begin{abstract}
abstract text (10 lines)
\end{abstract}
\tableofcontents{}

\newpage

\section{Introduction}

\paragraph{Paragraph 1 }

paragraph text

(ESN)

\section{Data}

\subsection*{MIDI-File (not implemented)}

\subsection*{Generation}

- generation which contains the same properties of the signals, but
it is generated instead of using midi files. 

\section{Methods}


\subsection{Echo State Network}

In this music generating task an ESN is used. The input signal of
the ESN is created as described in the pre-prosessing. The set up
of this ESN is based on `` GUIDE REFERECE ``. This guide contained
the base code of this project. That code has been rewritten to fit
this project. The transformers, sparce matrix and noice vector were
added as an addition. The documentation of the code can be found {*}here{*}.
Regarding the ESN, the following parameters have been set:

\#Maybe add the system equations etc

\subsubsection*{The resevoir size}

Within ESNs, it is serverly important that the resevoir is big enough,
such that is it possible to obtain the target output $y^{target}(n)$
from a linear combination of this signal space. The resevoir in this
project has been set to ...

\subsubsection*{The resevoir density }

The density of a resevoir is mainly dependent on the distribution
of the nonzero elements in the resevoir. In this project a basic uniform
distribution is used. Besides the distribution, the density of the
resevoir is set to .... 

\subsubsection*{Spectral radius}

Another main parameter for fitting the \noun{ESN} is the spectral
radius $\varrho$. This spectral radius is a parameter that will scale
the resevoir matrix $\mathbf{W}$. The effect this parameter has is
mainly seen on the learning accuracy of the the network. The new scaled
matrix $\mathbf{W}$ is calculated using 
\[
\boldsymbol{W{\scriptstyle new}}=\mathbf{W}*(\frac{\varrho}{\max(|\lambda|)})
\]

where $\lambda$ represents the eigenvalues of the resevoir matrix
$\mathbf{W}$, and $\mathbf{W_{new}}$ represents the updated resevoir
matrix. After experimenting, the $\varrho$ has been set to ...

\subsubsection*{Leaking rate}

The leaking rate $\alpha$ for an ESN determines how well a resevoir
unit maintains its value and how much it gets updated. Therefore $\alpha$
is one of the main parameters regarding the training process of this
project. In this project $\alpha$ is set to ...

\subsubsection*{Ridge regression}

Ridge regression (parameter). 

\subsubsection*{Transformers}

Transformer (the 3 mentioned is the intro paper: treshold, sigmoid,
sigmoid probability) (?) each transformer has a parameter and a squeezing
function (for now). 

\subsubsection*{Noice vector}

noise vector 

\subsection{Post-processing}

\subsubsection*{Output}

not sure what to write here yet 

\subsection{Fitting}

evaluation function (MLP/Jaeger idea). expanding parameters: bfs with
gradient decent.

\section{Results}

\section{Discussion}
\end{document}
