%% Do not edit unless you really know what you are doing.
\documentclass{extarticle}
\usepackage[utf8]{inputenc}
\usepackage{amsmath}
\usepackage{amsthm}
\usepackage[authoryear]{natbib}
\usepackage{graphicx}
\usepackage{geometry}
\usepackage{fancyvrb}

\makeatletter

\numberwithin{equation}{section}
\numberwithin{figure}{section}

\makeatother

\begin{document}
\begin{titlepage}
\begin{center}        
\vspace*{1cm}
       \textbf{ \huge Neural Networks Paper Title\\}
       \vspace{0.5cm}         Thesis Subtitle    \\                 
	   \vspace{1.5cm}
       \textbf{Sneha Lodha \\ Marco Gallo \\ Max Falziri \\ Yasmin de Groot}
       \vfill                                        
\vspace{0.8cm}              
                     
Neural Networks\\        
University of Groningen\\                
7th of July 2020                 
\end{center}
\end{titlepage}

\begin{abstract}
abstract text (10 lines)
\end{abstract}

\tableofcontents{}
\newpage


\section{Introduction}

\paragraph{} Classical music generation outside the domain of AI is mainly inspired by creativity and the knowledge of some musical elements.
There are several methods that have been used in the past to perform similar tasks, some of which include the use of algorithmic composition and Markov chains. %Ref?% 
Composing music through algorithmic rules can be restrictive in terms of originality, and user feedback. [1] Once the algorithms to generate the music 
have been defined, there is little to modify or fix. %%this sentence is weird%% 
When using Markov chains for the generation of music, although the presented output has more variation, 
drawbacks such as the violation of the Markov property come hand in hand. [2] This is mainly due to the fact that Markov property relies on simply the 
previous output which is not enough when it comes to generating musical patterns.  In this project, we will be attempting to generate musical drum 
patterns using an Echo State Network, also referred to as an ESN, and measure how close the generated pattern is to actual music.

\paragraph{} An ESN is a type of Recurrent Neural Network (RNN) with an input layer, a hidden layer, also known as a reservoir and an output layer. The interesting 
aspect of the ESN is that the reservoir to output weights are the only ones that need to be learned, and the weights of the reservoir itself are fixed. 
Due to this property of the learning the ESN is considered faster than some other RNNs.
We decided to use  

\section{Data}

\paragraph{} In this section the data used in this project is described and explained. The first subsection will explain how the drums are represented, and the second 
subsection will describe the generation of the music.

\subsection{Representation of the drums}

\paragraph{} The drum beats are represented in the network in the form of a matrix. In Table 1 an example of such a matrix can be seen. 
All the matrices will have 9 columns, each representing different notes as seen in the table.
Each row in the matrix this matrix will represent a quarter, and 4 row will represent a bar. If the tempo of the song is 
for example in 8th, 8 rows will together represent a bar. This is thus dependent on the tempo of the song. 

\paragraph{} The matrix seen in Table 1 represents 1 bar of music. Longer files will consist of more rows. Therefore the time is thus represented
in the rows of the matrix. In this table all the column are set to 0 instead of the bass drum. This will therefore produce one bar of
music with a bass drum playing on the first and third quarter. 



\begin{table}[h!]
       \centering
       \resizebox{\columnwidth}{!}{%
       \begin{tabular}{l|lllllllll}
       \cline{2-10}
                                       & Hi-hat-closed & Hi-hat-open & Bass drum & Crash & Snare & High-tom & Mid-tom & Floor-tom & Ride \\ \hline
       \multicolumn{1}{|l|}{Quarter 1} & 0             & 0           & 1         & 0     & 0     & 0        & 0       & 0         & 0    \\
       \multicolumn{1}{|l|}{Quarter 2} & 0             & 0           & 0         & 0     & 0     & 0        & 0       & 0         & 0    \\
       \multicolumn{1}{|l|}{Quarter 3} & 0             & 0           & 1         & 0     & 0     & 0        & 0       & 0         & 0    \\
       \multicolumn{1}{|l|}{Quarter 4} & 0             & 0           & 0         & 0     & 0     & 0        & 0       & 0         & 0   
       \end{tabular}%
       }
       \caption{An example of how the input matrix is structured.}
       \end{table}

\subsection{Generation of the music}

\paragraph{} The database used for training is generated by hand in this project. 
In order te generate music, two functions were implemented. The first one being \Verb"note_replicator" and the second one being \Verb"note_generator".
The replicator generates a note in a bar using the tempo, the note, and in what quarter it plays (this can also be a list of quarters / eight's etc). 
This bar is then repeated during the course of the whole dataset. 
The generated works in a similar way, only will this also need a variable representing how many bars this note is played. 
Besides the two generating function, another function is implements that allows the user to concatenate matrices. This is used to combine different "songs" 
into one long song in such a way that the network can use it to train. This way ... bars of music were generated to train the network.
% And Sneha explains a bit more. 

\section{Methods}

\paragraph{} In this experiment, one pipeline is used. This consists of the learning
algorithm, the post processing and the performance metrics. All of these steps will
be explained in detail in this section. The reason there is no pre-process present
is because the data is generated in such a way that it does not need to be pre-processed 
for the network. The way this data is generated was explained in the previous section. 
The code of this project is mainly written in Python, while the statistical analysis 
is done in R. 

\subsection{Echo State Network}

\paragraph{} In this music generating task an ESN is used. The input signal of
the ESN is created as described in the pre-processing. The set up
of this ESN is based on ``GUIDE REFERENCE``. This guide contained
the base code of this project. It also provided the needed knowledge about the 
network and its parameters as described below. The given code has been rewritten 
to fit this project. The transformers, sparse matrix and noise vector were
added as an addition. The documentation of the code can be found {*}here{*}.
Regarding the ESN, the following parameters have been set:

%% Maybe add the system equations etc %%

\subsubsection*{The reservoir size}

\paragraph{} Within ESNs, it is severely important that the reservoir is big enough,
such that is it possible to obtain the target output $y^{target}(n)$
from a linear combination of this signal space. The reservoir in this
project has been set to ...

\subsubsection*{The reservoir density}

\paragraph{} The density of a reservoir is mainly dependent on the distribution
of the nonzero elements in the reservoir. In this project a basic uniform
distribution is used. Besides the distribution, the density of the
reservoir is set to .... 

\subsubsection*{Spectral radius}

\paragraph{} Another main parameter for fitting the ESN is the spectral
radius $\varrho$. This spectral radius is a parameter that will scale
the reservoir matrix $\mathbf{W}$. The effect this parameter has is
mainly seen on the learning accuracy of the the network. The new scaled
matrix $\mathbf{W}$ is calculated using 
\[
\boldsymbol{W{\scriptstyle new}}=\mathbf{W}*(\frac{\varrho}{\max(|\lambda|)})
\]
where $\lambda$ represents the eigenvalues of the reservoir matrix
$\mathbf{W}$, and $\mathbf{W_{new}}$ represents the updated reservoir
matrix. After experimenting, the $\varrho$ has been set to ...

\subsubsection*{Leaking rate}

\paragraph{} The leaking rate $\alpha$ for an ESN determines how well a reservoir
unit maintains its value and how much it gets updated. Therefore $\alpha$
is one of the main parameters regarding the training process of this
project. In this project $\alpha$ is set to ...

\subsubsection*{Regularization}

\paragraph{} The regularization of this project is implemented using a ridge regression 
in the learning step of the network. This is used to stabilize the output 
in the long run. The parameter that scales the identity matrix in the ridge
regression is the actual set parameter in this situation. In this code this 
parameter is set to $e^{-8}$.

\paragraph{} Besides the ridge regression, a noise vector is also present in the code.
This is not used during this project. 

\subsubsection*{Transformers}

@Max :)
Transformer (the 3 mentioned is the intro paper: treshold, sigmoid,
sigmoid probability) (?) each transformer has a parameter and a squeezing
function (for now). 

\subsection{Post-processing}

\subsubsection*{Output}

not sure what to write here yet 

\subsection{Fitting}

evaluation function (MLP/Jaeger idea). expanding parameters: bfs with
gradient decent.

\section{Results}

\section{Discussion}

%% add bib %%
\end{document}
